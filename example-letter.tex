\DocumentMetadata{tagging=on,lang=en-US}
\documentclass[12pt]{hoflet}

\recipient{Barbara McClintock}
\salutation{Dr.~McClintock}
\mailingaddress{10 Cold Spring Harbor Laboratory\\ Cold Spring, NY 11500}
\myfullname{Christopher H. Eliot}
\myposition{Associate Professor}

\begin{document}

I am writing to recommend with utmost enthusiasm the distinguished ecologist Lucy Braun (deceased) for a position in your department. The following is from Wikipedia.

Emma Lucy Braun was born on April 19, 1889, in Cincinnati; she lived in Ohio for the remainder of her life. The daughter of George Frederick and Emma Moriah (Wright) Braun, her early interest in the natural world was encouraged by her parents, who took her and her older sister Annette Frances Braun into the woods to identify wildflowers. Braun's mother even had a small herbarium. 

In high school, Braun herself began collecting plants for study, the beginning of a huge personal herbarium that she assembled over her lifetime, composed of 11,891 specimens. Her collection became a part of the herbarium at the Smithsonian National Museum of Natural History in Washington D.C.

Braun studied botany and geology at the University of Cincinnati. She earned a bachelor's degree in 1910, a master's degree in geology in 1912, and a PhD in botany in 1914. In 1912, she studied with Henry C. Cowles; Harris M. Benedict was her dissertation adviser, with additional advice from Nevin M. Fenneman. She became the sixth woman to earn a PhD from that institution; her sister was the first. 

Braun's teaching and research career at the University of Cincinnati began as an assistant in geology (1910–1913). She taught as an assistant in botany from 1914 to 1917, then advanced through the titles of instructor, assistant professor, and associate professor. She achieved full professorship as a professor of plant ecology in 1946, two years before her retirement. She held the title of professor emeritus of plant ecology from 1948 until her death in 1971.

Braun was especially active in fieldwork, both during her active professorship as well as in retirement. She traveled over 65,000 miles in 25 years of investigations, most of it driving her own car. In addition to research nearby in Adams County, Ohio, and more widely in the east, Braun made 13 trips to the western United States. She was assisted in her work by her sister Annette, an entomologist and authority on Microlepidoptera.

Braun took numerous color photographs of the flora she encountered in her fieldwork, and displayed them as slides to illustrate her very popular lectures, both to university classes and the general public. In the hills of Kentucky during the period of Prohibition, Braun and her sister sometimes explored areas where moonshining was active; however, they maintained the trust of the local inhabitants, honoring local customs and not reporting illegal stills to authorities.
\end{document}
